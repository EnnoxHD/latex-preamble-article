% \PassOptionsToPackage{ngerman,}{babel} % use this for LyX compatibility where the babel package may detect an option clash
\documentclass[a4paper,oneside,11pt,bibtotoc,bibliography=openstyle]{scrartcl}
\usepackage{scrhack}
\usepackage[utf8]{inputenc}
\usepackage[T1]{fontenc}
\usepackage[ngerman]{babel}
\usepackage{lmodern}
\usepackage{amsfonts}
\usepackage{amssymb}
\usepackage{amsmath}
\usepackage{amsthm}
\usepackage{dsfont}
\usepackage{mathtools}
\usepackage{geometry}
\usepackage{graphicx}
\usepackage{svg}
\usepackage{pdfpages}
\usepackage{float}
\usepackage{caption}
\usepackage{longtable}
\usepackage[hang,flushmargin,multiple]{footmisc}
\setlength\parindent{0pt}
\usepackage[breaklinks]{hyperref}
\usepackage{xcolor}
\definecolor{darkred}{rgb}{0.6, 0, 0}
\definecolor{darkblue}{rgb}{0, 0, 0.6}
\hypersetup{
colorlinks,
citecolor=darkred,
linkcolor=darkred,
urlcolor=darkblue
}
\usepackage{cleveref}
\usepackage{caption}
\usepackage[newfloat]{minted}
\setminted{frame=lines,framesep=2mm,baselinestretch=1.2,fontsize=\footnotesize,linenos}
\setlength{\abovecaptionskip}{10pt}
\setlength{\belowcaptionskip}{15pt}
\usepackage{csquotes}
\usepackage{soulutf8}
\usepackage{ragged2e}
\usepackage{enumitem}
\usepackage{array}
\newcolumntype{L}[1]{>{\raggedright\let\newline\\\arraybackslash\hspace{0pt}}m{#1}}
\newcolumntype{C}[1]{>{\centering\let\newline\\\arraybackslash\hspace{0pt}}m{#1}}
\newcolumntype{R}[1]{>{\raggedleft\let\newline\\\arraybackslash\hspace{0pt}}m{#1}}
\usepackage[onehalfspacing]{setspace}
\usepackage[backend=biber,style=alphabetic]{biblatex}
\setcounter{biburlnumpenalty}{100}
\setcounter{biburlucpenalty}{100}
\setcounter{biburllcpenalty}{100}
\addbibresource{quellen.bib}
\usepackage{blindtext} % test text

\title{Titel}
\author{Autor}
\date{\today}

\begin{document}
\clearpage
\maketitle
\thispagestyle{empty}

\newpage
\tableofcontents

\newpage
\section{Einleitung}
\blindtext

\section{Hauptteil}
\blindtext\footnote{vgl. \cite{heise}}\\
\enquote{Zitat oder ähnliches}

\subsection{Erkenntnisse}

\subsubsection{A}

\subsubsection{B}

\subsubsection{Zusammenfassung}
\begin{itemize}
    \item Punkt A
    \item Punkt B
\end{itemize}

\section{Schluss}

Beispiel eines Java-Codeblocks:

\begin{minted}{java}
package example;

/**
  * point in 2D space
  */
public class Point2D {
    private int x;
    private int y;

    /**
      * private contructor hence we have to set the coords explicit
      */
    private Point2D() { }

    public Point2D(int x, int y) {
        setX(x);
        setY(y);
    }

    public int getX() {
        return this.x;
    }

    public void setX(int x) {
        this.x = x;
    }

    public int getY() {
        return this.y;
    }

    public void setY(int y) {
        this.y = y;
    }
}
\end{minted}

\newpage
\printbibliography

\newpage
\blindtext\\\\

\begin{tabular}{L{5.5cm} C{0.5cm} L{7.5cm}}
     & & \\
    \cline{1-1}\cline{3-3}
    \texttt{Ort, Datum} & & \texttt{Unterschrift}
\end{tabular}

\end{document}
